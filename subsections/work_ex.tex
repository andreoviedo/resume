\section{Professional Experience}
\subsection{Economic Research Analyst}{Superintendency of Banks of Peru (SBS)}{January 2022 - August 2023}
%\vspace{0.1cm}
\begin{itemize}
    %\item Reviewed banking regulation papers seeking to align Peru banking standards to Basel proposals and global best practices
    %\item First review of global standards for a later-published paper on climate risk stress testing on the Peruvian financial system
    
    \item Spearheaded the development of the new IFRS 9 based loan provisions framework for the Peruvian financial system (>100 billion USD), leveraging R and SQL to calibrate provision requirements, managing datasets exceeding 100 million rows
    \item Developed a Python script to webscrap and organize historical information of the pensions' holdings, reducing the time required for the project from 3 months to 8 hours; information used as main input for a research paper
    %\item Presented new loan provisions framework impact analyses to senior executives and task forces, impacting system-wide decisions for 64 financial entities and 8 million debtors
    \item Engineered a macroprudential heatmap pipeline integrating 60 local and global indicators, reducing data update time from two days to one minute—an improvement of 99.97\% in processing time by streamlining API connections using R and SQL
    \item Lead the risk analysis for \href{https://www.sbs.gob.pe/noticia/detallenoticia/idnoticia/2680}{the new definition for medium-sized enterprises} calibrated on default rate data and novel information on sales tax; proposal incorporated to Peruvian banking standards
    \item Designed executive reports on the ex-post impact of government-issued Covid-19 payment protection programs and potential portfolio risks of El Niño floodings on the Peruvian financial system
    %\item Led the creation of an analytics platform leveraging the use of Python, R and SQL, allowing reliable and organized access to the department's reports by executive teams
    %\item Presented to senior executive teams the potential effect of El Niño on the financial system leveraging 

\end{itemize}
%\vspace{0.1cm}
\subsection{Credit Risk Supervisor}{Superintendency of Banks of Peru (SBS)}{May 2019 - December 2021}
%\vspace{0.1cm}
\begin{itemize}
    %\item Reviewed literature on multiple machine learning methodologies to build a comprehensive guide for senior management
    \item Developed a system-wide default rate calculation pipeline in R running parallel SQL processes, cutting processing time by 75\% while generating 250 more output; visualized results using Shiny and PowerBI
    \item In charge of re-designing the system-wide loan provisions analysis script in R, reducing processing times by 98.6\%
    \item Designed a Covid-19 payment protection program (+15 B PEN) analysis pipeline, reducing processing times by 87\%  
    
    

    

\end{itemize}
%\vspace{0.1cm}
\subsection{Financial System Supervisor}{Superintendency of Banks of Peru (SBS)}{May 2018 - May 2019}
%\vspace{0.1cm}
\begin{itemize}
    \item Pioneered the use of open-source statistical software by building an R package for Oracle database connection; mentored junior and senior analysts on its usage, development and maintenance;
    \item Regulated the implementation of an XGBoost income predictor model for Peru’s largest bank, covering over 9 million potential debtors, the first large-scale machine learning income predictor in the country
    \item Developed a comprehensive data ecosystem and a PowerBI dashboard for the credit card market (> 10 million credit cards) analyzing profitability, client behavior, and market trends, among other KPIs
    


    %\item Enhanced the department calculation of its own tailor-made default rate series by using SQL and R
    
\end{itemize}


